\makeatletter
\newcommand*{\figref}[1]{{\if@english Figure~\else 図\fi\ref{#1}}}
\newcommand*{\tabref}[1]{{\if@english Table~\else 表\fi\ref{#1}}}
\newcommand*{\eqref}[1]{{\if@english Equation~\else 式\fi}(\ref{#1})}
\newcommand*{\chapref}[1]{{\if@english Chapter~\ref{#1}\else 第{\ref{#1}}章\fi}}
\newcommand*{\secref}[1]{{\if@english Section~\ref{#1}\else \ref{#1}節\fi}}

\newcommand{\chapprint}[1]{\chapref{#1}\if@english:~\textit{\nameref{#1}}\else「\textit{\nameref{#1}}」\fi}
\newcommand{\secprint}[1]{\secref{#1}\if@english:~\textit{\nameref{#1}}\else「\textit{\nameref{#1}}」\fi}

\def\PDFTITLE{\if@english \@ethesisname \else \@jthesisname\fi}
\def\PDFSUBJECT{\if@english \@etitle \else \@jtitle\fi}
\def\PDFAUTHOR{\if@english \@eauthor \else \@jauthor\fi}
\def\PDFKEYWORDS{\ifdefined\@keywords\@keywords\fi}

\makeatother

\usepackage[dvipdfmx]{color}
\usepackage[dvipdfmx,bookmarks=true,bookmarksnumbered=true,bookmarkstype=toc,
     colorlinks=true,
     linkcolor=black,
     citecolor=black,
     urlcolor=black,
     pdftitle=\PDFTITLE,
     pdfsubject=\PDFSUBJECT,
     pdfauthor=\PDFAUTHOR,
     pdfkeywords=\PDFKEYWORDS]{hyperref}
\usepackage{pxjahyper} % for japanese encoding on pdf bookmarks
\usepackage{listings}

\lstset{
  breaklines = true,
  basicstyle=\ttfamily,
  commentstyle={\itshape \color[cmyk]{1,0.4,1,0}},
  classoffset=1,
  %% keywordstyle={\bfseries \color[cmyk]{0,1,0,0}},
  %% stringstyle={\ttfamily \color[rgb]{0,0,1}},
  showstringspaces=false,
  columns=fullflexible,
  keepspaces=true,
  escapeinside={<@}{@>},
  %% frame=tRBl,
  framesep=5pt,
  %% numbers=left,
  stepnumber=1,
  numberstyle=\tiny,
  tabsize=2,
}
